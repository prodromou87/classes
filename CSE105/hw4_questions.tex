\documentclass[letterpaper,12pt]{article}
\usepackage{fullpage}
\usepackage{amssymb}
\usepackage{mathtools}
\usepackage[hyphens]{url}
\usepackage[breaklinks,linkcolor=black,citecolor=black,urlcolor=black,bookmarks=true,bookmarksopen=false,pdftex]{hyperref}
\usepackage{graphicx}
\usepackage{color}
\usepackage[protrusion=true,expansion=true,kerning]{microtype}

\setlength{\topmargin}{0in}
\setlength{\headheight}{0in}
\setlength{\headsep}{0in}
\setlength{\topskip}{0in}

\def\dash---{\kern.16667em---\penalty\exhyphenpenalty\hskip.16667em\relax}

\newcommand{\rev}{^{\scriptscriptstyle\mathcal{R}}}
\newcommand{\spc}{\textvisiblespace}
\newcommand{\yields}{\Rightarrow}
\newcommand{\derives}{\stackrel{*}{\Rightarrow}}

\begin{document}

\newlength{\boxwidth}
\setlength{\boxwidth}{\textwidth}
\addtolength{\boxwidth}{-2cm}
\noindent\framebox[\textwidth]{\hfil
\parbox[t]{\boxwidth}{%
{\bfseries CSE\,105: Automata and Computability Theory \hfill Autumn 2014}
\begin{center}\huge Homework \#4\end{center}
Due: Thursday, December 11th, 2014%
}\hfil}
\vspace{0.7cm}

\thispagestyle{empty}

\begin{description}

\item[Problem 1] Show that the class of decidable (i.e., recursive)
  languages is closed under Kleene star.

  \textbf{Hint:} There's an elegant algorithm, using dynamic
  programming, that makes $O(n^2)$ oracle calls when checking a string
  of length~$n$; you might wish to describe this algorithm instead of
  the na{\"\i}ve exponential one.

\item[Problem 2] Show that the class of R.E.\ (i.e., recognizable)
  languages is closed under intersection.

\item[Problem 3] Consider a variant definition of Turing machines,
  called \emph{always-right} Turing machines.  In such a machine, the
  transition function $\delta\colon Q \times \Gamma \to Q \times
  \Gamma \times \{R\}$ specifies that the machine always moves its
  head right (``$R$'').  There is no way for a machine in the
  always-right model to move its head left.

  Show that the always-right Turing machine model is \emph{not}
  equivalent to the Turing machine model defined in class and in
  Sipser.

  \textbf{Hint:} Always-right Turing machines are, in fact, equivalent
  in power to DFAs.

\item[Problem 4] Let $L_4$~be the language
  \begin{displaymath}
    \Bigl\{
        \langle M,w \rangle
      \Bigm|
      \parbox{3.5in}{$M$~is a Turing machine, $w$~is a string,
        and $M$ never moves its head left when run on input~$w$.}
      \Bigr\} \enspace.
  \end{displaymath}
  Show that $L_4$ is decidable (i.e., recursive), by describing a
  Turing machine that decides~$L_3$.

  (Note that the Turing machines here are the ordinary Turing machines
  defined in class and in Sipser, not the always-stay machines of
  problem~3, and their transition function is allowed to specify a
  move left.)

\item[Problem 5] Let $L_5$~be the language
  \begin{displaymath}
    \bigl\{
        \langle G, D \rangle
      \bigm|
      \text{$G$~is a CFG, D is a DFA, and
        $L(D) \subseteq L(G)$.}
      \bigr\} \enspace.
  \end{displaymath}
  Show that $L_5$ is undecidable.

  Is $L_5$ R.E.?  Is it co-R.E.?

\end{description}
\end{document}
