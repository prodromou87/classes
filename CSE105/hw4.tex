\documentclass[letterpaper,12pt]{article}
\usepackage{fullpage}
\usepackage{amssymb}
\usepackage{mathtools}
\usepackage[hyphens]{url}
\usepackage[breaklinks,linkcolor=black,citecolor=black,urlcolor=black,bookmarks=true,bookmarksopen=false,pdftex]{hyperref}
\usepackage{graphicx}
\usepackage{color}
\usepackage[protrusion=true,expansion=true,kerning]{microtype}

\setlength{\topmargin}{0in}
\setlength{\headheight}{0in}
\setlength{\headsep}{0in}
\setlength{\topskip}{0in}

\newcommand{\Z}{\mathbb{Z}}
\newcommand{\F}{\mathbb{F}}
\newcommand{\R}{\mathbb{R}}

\def\dash---{\kern.16667em---\penalty\exhyphenpenalty\hskip.16667em\relax}

% TODO: FILL IN NAME, ID, AND COLLABORATORS HERE
\def\name{ANDREAS PRODROMOU}
\def\id{A53049230}
\def\collabs{NONE}
% END PART TO FILL IN


\begin{document}

\newlength{\boxwidth}
\setlength{\boxwidth}{\textwidth}
\addtolength{\boxwidth}{-2cm}
\noindent\framebox[\textwidth]{\hfil
\parbox[t]{\boxwidth}{%
{\bfseries CSE\,105: Automata and Computability Theory \hfill Autumn 2014}
\begin{center}\huge Homework \#4 Solutions\end{center}
\name\hfill\id%
}\hfil}
\vspace{0.7cm}

\noindent\textbf{Collaborated with:} \collabs
\bigskip


\noindent\textbf{Solution to Problem 1}\\
Assume that language A is decidable. This means $\exists$ TM M which decides A. For all $w \in \Sigma^*$, if $w \in A$ then M(w) accepts, otherwise M(w) rejects.

By definition, a string W $\in$ A iff W = $\varepsilon$ or W = $X_1X_2\ldots X_k$ for some k and $X_{1\ldots k}$ are non-empty and belong in A. To show that $A^*$ is decidable, I will define a new TM N that decides $A^*$. This means that for a given string W $\in \Sigma^*$, N(W) accepts if $W \in A*$ and rejects otherwise.\\

N = On input W:\\
\indent\indent If W = $\varepsilon$ accept\\
\indent\indent Let n be the length of W\\
\indent\indent For k = 1 to n do\\
\indent\indent\indent For each possible $X_1,X_2,\ldots,X_k \in \Sigma^* - \{\varepsilon\}$, where $W = X_1X_2\ldots X_k$ do\\
\indent\indent\indent\indent Run $M(X_i)$ for i = 1 to k\\
\indent\indent\indent\indent If all computation accept, then accept\\
\indent\indent\indent\indent If any computation rejects, reject\\

\noindent\textbf{Solution to Problem 2}\\ 
Assume languages A and B are recognizable and $M_A$ and $M_B$ are the TMs that recognize them. I will define a new TM N that recognizes the intersection of A and B.\\

N = On input w:\\
\indent\indent Run $M_A$ on w. If it rejects, reject.\\
\indent\indent If $M_A$ accepts w, Run $M_B$ on w. If it rejects, reject. If it accepts, accept.\\

The new machine will accept a string iff $M_A$ accepts it first and then $M_B$ accepts it too. Consequently, N recognizes the language $A \cap B$.\\

\noindent\textbf{Solution to Problem 3}\\
We can observe that we can simulate a DFA using an always-right TM. We simply need to add transitions from the states in F to $q_{accept}$ when blank is read and transitions from states outside of F to $q_{reject}$ when blank is read. Consequently, DFAs, are a subset of always-right TMs.

Now assume a TM M = $(Q, \Sigma, \Gamma, \delta, q_0, q_{accept}, q_{reject})$. I will construct a DFA D = $(Q',\Sigma',\delta', q_0', F)$ that recognizes the same language, where:
$ Q' = Q $, $\Sigma' = \Sigma$ and $q_0' = q_0$. The transition function will be:

\begin{equation}
\delta'(q,s) = \left\{ 
  \begin{array}{l l}
    q & \quad \text{if q $\in \{q_{accept}, q_{reject}\}$ }\\
    q' & \quad \text{where M has transition (q,s) to q'}
  \end{array} \right.
\end{equation}

F has to be defined as the set containing $q_{accept}$ and the set of states q $\in$ Q, q $\ne q_{accept}, q_{reject}$, suche that M starting at q and reading blanks, eventually enters $q_{accept}$.

I showed that DFAs are a subset of always-right TMs and that always-right TMs are a subset of DFAs. Consequently, always-right TMs are equivalent to DFAs.\\

\noindent\textbf{Solution to Problem 4}\\  
In a machine that can never move left, we can observe that any symbols written by this machine will never be read again. I will define a new character which is not in $\Gamma$, which will be written on every transition. If that character is read at some state, I will know that the machine moved left and I will transition to the $q_{reject}$ state. The new TM will be the same as the old, except the new symbol and the simple transition rule upon reading this symbol.

This new machine is a decider for $L_4$, since it will either accept if M accepts and only moves right, and it will reject if M moves left or attempts to loop.


\end{document}
