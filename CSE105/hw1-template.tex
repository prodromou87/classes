\documentclass[letterpaper,12pt]{article}
\usepackage{fullpage}
\usepackage{amssymb}
\usepackage{mathtools}
\usepackage[hyphens]{url}
\usepackage[breaklinks,linkcolor=black,citecolor=black,urlcolor=black,bookmarks=true,bookmarksopen=false,pdftex]{hyperref}
\usepackage{graphicx}
\usepackage{color}
\usepackage[protrusion=true,expansion=true,kerning]{microtype}

\setlength{\topmargin}{0in}
\setlength{\headheight}{0in}
\setlength{\headsep}{0in}
\setlength{\topskip}{0in}

\newcommand{\Z}{\mathbb{Z}}
\newcommand{\F}{\mathbb{F}}
\newcommand{\R}{\mathbb{R}}

\def\dash---{\kern.16667em---\penalty\exhyphenpenalty\hskip.16667em\relax}

% TODO: FILL IN NAME, ID, AND COLLABORATORS HERE
\def\name{ANDREAS PRODROMOU}
\def\id{A53049230}
\def\collabs{NONE}
% END PART TO FILL IN


\begin{document}

\newlength{\boxwidth}
\setlength{\boxwidth}{\textwidth}
\addtolength{\boxwidth}{-2cm}
\noindent\framebox[\textwidth]{\hfil
\parbox[t]{\boxwidth}{%
{\bfseries CSE\,105: Automata and Computability Theory \hfill Autumn 2014}
\begin{center}\huge Homework \#1 Solutions\end{center}
\name\hfill\id%
}\hfil}
\vspace{0.7cm}

\noindent\textbf{Collaborated with:} \collabs
\bigskip


\noindent\textbf{Solution to Problem 2}\\

% PROBLEM 2 SOLUTION GOES HERE
\textbf{a.} Since $a \in P(m)$, it can be written as $a = im$, where $i \in \Z$. Similarly, since $b \in P(m)$, it can be written as $b = jm$, where $j \in \Z$. Consequently, $a + b$ can be written as $im + jm$ which is equal to $(i + j)m$. Since the set $\Z$ is closed under addition, $i+j \in \Z$. Thus , $P(m)$ is closed under addition since $(i + j)m \in \Z \implies a+b \in \Z$.\\

\textbf{b.} Similarly to (a), since $a,b \in P(m)$, they can be written as $a = im$, and $b = jm$ respectively, where $i,j \in \Z$. Consequently, $ab = imb$. Since $b = jm$ and $j,m \in \Z \implies b \in \Z$. Thus, $ib \in \Z$ and $(ib)m \in P(m) \implies ab \in P(m)$.\\

\textbf{c.} By the definition of A and B we know that $A+B = \{a+b \mid a \in A, b \in B\} = \{im + jn \mid i,j \in \Z, m,n \in \Z^+\}$. Let's assume that $r \in \Z$ is the gcd of m and n. This means that $m = cr$ and $n = kr$, where $c,k \in \Z$. Consequently, $A+B$ can be written as $A+B = \{icr + jkr \mid i,j,c,k \in \Z, r \in \Z^+\} = \{(ic+jk)r \mid i,j,c,k \in \Z, r \in \Z^+\}$. And since $(ic+jk) \in \Z$ ($\Z$ is closed under addition), $A+B = P(r)$.

\end{document}
